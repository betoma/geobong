\chapter[Historical \& Cultural Context][Historical \& Cultural Context of \oldlang{}]{Historical \& Cultural Context}

Very little is known about goblin culture when \oldlang{} was spoken outside of the legends and mythology of later goblins. While there are written works in this language, it is often unclear what is fictional and what is historical. As a result, what follows should be taken with a grain of salt.

\todo[inline]{write the rest of this lol}


\chapter[Phonology][Phonology of \oldlang{}]{Phonology of \oldlang{}}

We cannot know with 100\% certainty how exactly the phonemes in \oldlang{} were realized. However, based on descriptions of the language in historical goblin writings, it is largely possible to construct a simple phonology of the language during the height of the Goblin Golden Age, albeit with only approximations of the actual phonetics. In addition, the sound changes that have taken place since this period of Goblin history are remarkably well-evidenced, allowing a for fairly clear timeline of sound changes in \oldlang{} to be constructed.

\section{Golden Age \oldlang{}}

\subsection{Orthography}

\todo[inline]{Write stuff about the ortho. It's a logography.}

\subsection{Phonological Inventory}

The picture we have of \oldlang{}'s phonology in the Goblin Golden Age is that it had an extremely regular structure. However, it's impossible to say whether this accurately reflects the language as it was spoken at the time. It's certainly possible that there were irregularities at the time that were neither described in written works nor left any effects on future sound changes. Lacking evidence for that, however, we focus on the features of the language we have evidence for either through historical record or later sound changes.

\subsubsection{Consonants}

\begin{table}[hbt]
    \centering
    \begin{tabular}{@{}rccccc@{}}
        \toprule
        & Labial & Laminal & Apical & \multicolumn{1}{c}{Palatal} & Velar \\ \midrule
        Stop & p & t\lamino & t\apico &  & k \\
        Nasal & m & n\lamino & n\apico &  &  \\
        Approximant &  & \multicolumn{2}{c}{\alvr} & j &  \\
        Lateral Approx. &  & \multicolumn{2}{c}{l} &  &  \\ \bottomrule
    \end{tabular}
    \caption{\oldlang{} Consonant Inventory}
    \label{tab:old-consonants}
\end{table}

Modern \lang{} possesses many more consonant phonemes than its progenitor and makes several featural distinctions that were not made in \oldlang{} as it was spoken during the golden age. For instance, we have no evidence that a voicing distinction was made in \oldlang{} at this time, and the voicing distinction present in \lang{} can generally be attributed to a series of sound changes causing it to be phonemicized well after this period. However, there are some features that we can be certain were phonemically distinguished in Golden Age \oldlang{}, if not earlier.

\paragraph{Laminal/Apical Distinction}

The laminal/apical distinction in coronal occlusives is a particularly notable feature of \langfam{} languages, as it is a feature not shared with other language families present in \thecontinent{}. We have a clear contemporary description of this distinction in the Golden Age in a text dated to roughly 300 years prior to Elvish contact:

\begin{displayquote}[][]
    \textit{An outsider arrived in town asking to trade. He tried to speak \oldlang{} to impress us, but could only use the tip of his tongue. \lex{oldlang:goods} became \lex{oldlang:excrement} and \lex{oldlang:provide} became \lex{oldlang:eat}. No trade was done, but all were very much amused.}
\end{displayquote}

In addition to describing the particular failing of this outsider as \enquote{only us[ing] the tip of his tongue}, the above text also provides two clear minimal pairs. As a result, it's not particularly difficult to decipher the phonetic forms here, even despite \oldlang{}'s opaque orthography. \todo[inline]{Add modern reflexes of these words}

However, this is the extent of our knowledge about such a distinction. We have little evidence whether coronal approximants, for instance, maintained an apical/laminal distinction at this point in time. As a result, these can be reconstructed as either possessing such a distinction at the point in history, or developing such a distinction through sound change over time based on proximity to other coronals. Here we take the latter approach, as it appears all modern-day occurrences of this phonemic distinction in \lang{} are adequately explained by known sound changes.

\subsubsection{Vowels}

\subsection{Phonotactics \& Syllable Structure}

\section{Sound Changes Since the Golden Age}



\chapter[Morphosyntax][Morphosyntax of \oldlang{}]{Morphosyntax of \oldlang{}}